\documentclass{article}
\documentclass[format=sigconf]{acmart}
\usepackage[utf8]{inputenc}

\title{Multipath TCP}

\date{September 2020}

\begin{document}

\maketitle

\newpage
\section{Abstract}

This report focuses on investigate throughput optimization by using the Mutli-path TCP networkprotocol. As more and more wireless devices gets connected to the internet the demand for handling mobility increases when hosts moves from one place to another. Multihoming is a concept that means that a host or computer network can connect to more than one network. This enables throughput optimization if any network link should decrease or disappear. Multi-path TCP is built on the concept of multihoming and is used among other things for WIFI and mobile networks. In this report we will study the advantages and disadvantages of the Multi-path TCP protocol and with a experiment test its function.   



\section{Introduction}

The high growth rate of data traffic and wireless devices together with new technologies such as 5G puts higher demands on how to handle mobility and optimize throughput. 
The TCP protocol in the network layer uses only one path when establishing connection between two endpoints, this means that when connection is established, endpoints can not change. This leads to that a internet connection must be broken down when switching from for a example WIFI to 4G when the host moves from one place to another. This is a major disadvantage of conventional TCP. Most devices today use several interfaces to optimize throughput. With many interfaces the opportunity is created to use several internet connections at the same time and this concept is called multihoming.
Multi-path TCP is a network protocol designed to use multiple interfaces. The development of similar protocols is very important so throughput and reliability can increase as the wireless devices becomes more common.
This report will help highlight the importance of Multi-path TCP when more and more devices becomes wireless but also look at its shortcomings. 

\section{Time plan}









\end{document}
